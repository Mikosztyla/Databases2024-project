\UseRawInputEncoding
\documentclass[12pt]{article}
\usepackage[english]{babel}
\usepackage{listings}
\usepackage{caption}
\usepackage[a4paper, margin=1in]{geometry}
\usepackage{polski}
\usepackage[utf8]{inputenc}
\usepackage{hyperref}
\usepackage{float}
\usepackage{graphicx}
\usepackage{enumitem}

\title{\includegraphics[width=0.2\textwidth]{agh.jpg} \\ \textbf{MongoDB projekt}}
\author{Mikołaj Gosztyła, Michał Dydek}
\date{27.05.2024}

\usepackage{xcolor}
\usepackage{color}
\definecolor{lightgray}{rgb}{.9,.9,.9}
\definecolor{darkgray}{rgb}{.4,.4,.4}
\definecolor{purple}{rgb}{0.65, 0.12, 0.82}

\lstdefinelanguage{JavaScript}{
  keywords={typeof, new, true, false, catch, function, return, null, catch, switch, var, if, in, while, do, else, case, break},
  keywordstyle=\color{blue}\bfseries,
  ndkeywords={class, export, boolean, throw, implements, import, this},
  ndkeywordstyle=\color{darkgray}\bfseries,
  identifierstyle=\color{black},
  sensitive=false,
  comment=[l]{//},
  morecomment=[s]{/*}{*/},
  commentstyle=\color{purple}\ttfamily,
  stringstyle=\color{red}\ttfamily,
  morestring=[b]',
  morestring=[b]"
}

\lstset{
   language=JavaScript,
   backgroundcolor=\color{lightgray},
   extendedchars=true,
   basicstyle=\footnotesize\ttfamily,
   showstringspaces=false,
   showspaces=false,
   numbers=left,
   numberstyle=\footnotesize,
   numbersep=9pt,
   tabsize=2,
   breaklines=true,
   showtabs=false,
   captionpos=b
}



\addto\captionsenglish{\renewcommand{\contentsname}{Spis treści}}

\begin{document}
\maketitle

\hypersetup{
    linktoc=all
}
\tableofcontents

\newpage
\section{Opis zadania i technik do jego realizacji}
Aplikacja obsługuje restaurację i niektóre działania, które mogłyby być przydatne dla jej pracowników. Operacje, które wspiera nasze oprogramowanie to: 
\begin{itemize}
	\item rezerwacje - klient dzwoni do pracownika, który nastepnie wprowadza dane do systemu
	\item dodawanie wydatków - bieżące wydatki przez firmę mogą być wprowadzane przez pracownika
	\item dodawanie dochodów - po każdej transakcji pracownik również może wprowadzić dochody wraz z niektórymi informacjami odnośnie nich 
\end{itemize}

Zdecydowalismy się na stworzenie oprogramowania od strony pracowniczej, a nie dla klientów ponieważ jako pracownik mamy większą kontrolę nad wprowadzanymi danymi.

\subsection{Techniki wykorzystane do realizacji}
todo: screeny frontendu, backendu
\begin{itemize}
	\item Front-End - do realizacji front-endu wykorzystaliśmy język JavaScript, a także wykorzystaliśmy framework React
	\item Backend - backend realizuje skrypt napisany również w JavaScripcie przy użyciu frameworka Express, a do komunikacji z bazą danych został użyty framework Mongoose
	\item Baza danych - rodzaj bazy danych na jaką się zdecydowaliśmy to nierelacyjna baza MongoDB 
\end{itemize}

\newpage

\section{Kolekcje}
\begin{samepage}
\subsection{Kolekcja employees}
W tej kolekcji przechowujemy dane na temat każdego pracownika. Aplikacja umożliwia logowanie każdego z użytkowników w systemie. Dodatkowo można tworzyć nowe konta.
todo: zdjęcie danych
\begin{lstlisting}[caption={Employees}]
const employeeSchema = new mongoose.Schema({
    name: { type: String, required: true },
    surname: { type: String, required: true },
    employee_number: { type: Number, required: true, unique: true },
    password: { type: String, required: true }
});
\end{lstlisting}
\end{samepage}

\begin{itemize}
	\item \textbf{name} - imie pracownika
	\item \textbf{surname} - nazwisko pracownika
	\item \textbf{employee} number - numer pracownika
	\item \textbf{password} - hasło konta pracownika
\end{itemize}

\newpage
\begin{samepage}
\subsection{Kolekcja res}
W tej kolekcji znajdują się aktualne rezerwacje na stoliki w restauracji. Na tej kolekcji jest również nałożony dodatkowo trigger, który usuwa stare rezerwacje.
todo: zdjęcie danych
\begin{lstlisting}[caption={Reservations}]
const resSchema = new mongoose.Schema({
    employee_id: { type: String, required: true },
    date: { type: String, required: true },
    time: { type: String, required: true },
    duration: { type: Number, required: true },
    table: { type: String, required: true },
});
\end{lstlisting}
\end{samepage}

\begin{itemize}
	\item \textbf{employee id} - numer pracownika, który dokonał rezerwacji
	\item \textbf{date} - dzień, miesiąc i rok rezerwacji
	\item \textbf{time} number - godzina rozpoczęcia rezerwacji
	\item \textbf{duration} - długość rezerwacji
	\item \textbf{table} - numer stolika, którego dotyczy rezerwacja
\end{itemize}

\newpage
\begin{samepage}
\subsection{Kolekcja expenses}
Kolekcja, w której przechowywane są wydatki firmowe i każdy z pracowników ma możliwośc wprowadzenia danych, które są wykorzystywane w raporcie.
todo: zdjęcie danych
\begin{lstlisting}[caption={Expenses}]
const expenseSchema = new mongoose.Schema({
	employee_number: { type: Number, required: true },
	item: { type: String, required: true },
	quantity: { type: Number, required: true },
	unit_price: { type: Number, required: true },
	date: { type: String, required: true },
});
\end{lstlisting}
Jest to kolekcja, na której wykonywane są wszystkie operacje CRUD.
todo: opis cruda
\end{samepage}

\begin{itemize}
	\item \textbf{employee number} - numer pracownika, który zerejestrował wydatek
	\item \textbf{item} - rzecz, której dotyczy dodany wydatek
	\item \textbf{quantity} - ilość 
	\item \textbf{unit\_price} - cena jednostkowa
	\item \textbf{date} - data wydatku
\end{itemize}

\newpage
\begin{samepage}
\subsection{Kolekcja incomes}
Kolekcja, w której przechowywane są przychody dla firmy i każdy z pracowników ma również możliwośc wprowadzenia danych, które są później wykorzystywane w raporcie.
todo: zdjęcie danych
\begin{lstlisting}[caption={Incomes}]
const incomeSchema = new mongoose.Schema({
    employee_number: { type: Number, required: true },
    order_id: { type: String, required: true },
    price: { type: Number, required: true },
    date: { type: String, required: true },
});
\end{lstlisting}
Jest to kolekcja, na której wykonywane są wszystkie operacje CRUD.
todo: opis cruda
\end{samepage}

\begin{itemize}
	\item \textbf{employee number} - numer pracownika, który zerejestrował wpływ
	\item \textbf{order id} - rzecz, której dotyczy dodany wpływ
	\item \textbf{price} - cena
	\item \textbf{date} - data wpływu
\end{itemize}

\newpage
\begin{samepage}
\subsection{Kolekcja properties}
todo: zdjęcie danych
\begin{lstlisting}[caption={Properties}]
const propertiesSchema = new mongoose.Schema({
    numberOfTables: { type: Number, required: true },
    openingTime: { type: String, required: true },
    closingTime: { type: String, required: true },
    closedDays: [{ type: String, required: true }]
});
\end{lstlisting}
\end{samepage}

\begin{itemize}
	\item \textbf{numberOfTables} - ilość stolików w restauracji
	\item \textbf{openingTime} - godzina otwarcia restauracji
	\item \textbf{closingTime} - godzina zamknięcia restauracji
	\item \textbf{closedDays} - dni, w które restauracja będzie zamknięta
\end{itemize}

\newpage
\section{Endpointy}
\subsection{GET}
\begin{itemize}
    \item \textbf{\texttt{\textbackslash config}} - pobiera konfigurację restauracji, która znajduje się w folderze \texttt{Config} i pliku \texttt{restaurantProperties.json}
    \item \textbf{\texttt{\textbackslash employees}} - pobiera listę wszystkich pracowników, bez hasła należącego do nich konta
    \item \textbf{\texttt{\textbackslash reservations}} - pobiera listę wszystkich rezerwacji
    \item \textbf{\texttt{\textbackslash expensesList}} - pobiera listę wszystkich wydatków
    \item \textbf{\texttt{\textbackslash incomesList}} - pobiera listę wszystkich dochodów
\end{itemize}


\subsection{POST}
\begin{itemize}
    \item \textbf{\texttt{\textbackslash employeeAdd}} - dodaje nowego pracownika
    \item \textbf{\texttt{\textbackslash login}} - loguje pracownika na jego konto
    \item \textbf{\texttt{\textbackslash deleteAccount}} - usuwa konto pracownika
    \item \textbf{\texttt{\textbackslash res}} - dodaje nową rezerwację
    \item \textbf{\texttt{\textbackslash expense}} - dodaje nowy wydatek
    \item \textbf{\texttt{\textbackslash income}} - dodaje nowy dochód
    \item \textbf{\texttt{\textbackslash saveTurnoverData}} - zapisuje raport do pliku data.csv
\end{itemize}


\subsection{PUT}
\begin{itemize}
    \item \textbf{\texttt{\textbackslash expenses \textbackslash}} - aktualizuje wydatek o podanym ID
    \item \textbf{\texttt{\textbackslash incomes \textbackslash}} - aktualizuje dochód o podanym ID
\end{itemize}


\subsection{DELETE}
\begin{itemize}
    \item \textbf{\texttt{\textbackslash reservations \textbackslash}} - usuwa rezerwację o podanym ID
    \item \textbf{\texttt{\textbackslash expenses \textbackslash}} - usuwa wydatek o podanym ID
    \item \textbf{\texttt{\textbackslash incomes \textbackslash}} - usuwa dochód o podanym ID
\end{itemize}

\newpage
\section{Raport}
Raport, który umożliwia nasza aplikacja łączy 3 kolekcje: incomes, expenses, employees. Polega on na wygenerowaniu całkowitego obrotu i ilości dodanych rekordów do dwóch pierwszych kolekcji. Zdecydowaliśmy się na taki wybór, ponieważ uważamy, że jest to przydatna i potrzebna informacja dla pracodawcy, który pewnie chciałby mieć wgląd w niektóre dane konkretnych pracowników. Dodatkowo można uwzględnić konkretny rok i miesiąc, by wygenerować raport dla konkretnej daty.

\begin{lstlisting}[caption={Zapytanie do bazy danych realizujące raport}]
db.incomes.aggregate([
{
	$addFields: {
		year: { $year: { $toDate: "$date" } },
		month: { $month: { $toDate: "$date" } }
	}
},
{
	$project: {
		employee_number: 1,
		price: 1
	}
},
{
	$unionWith: {
		coll: "expenses",
		pipeline: [
		{
			$addFields: {
				year: { $year: { $toDate: "$date" } },
				month: { $month: { $toDate: "$date" } }
			}
		},
		{
			$project: {
				employee_number: 1,
				price: { $multiply: ["$unit_price", "$quantity"] }
			}
		}
		]
	}
},
{
	$group: {
		_id: "$employee_number",
		count: { $sum: 1 },
		monetary_turnover: { $sum: "$price" }
	}
},
{
	$sort: {
		count: -1
	}
},
{
	$lookup: {
		from: "employees",
		localField: "_id",
		foreignField: "employee_number",
		as: "employee_info"
	}
},
{
	$unwind: "$employee_info"
},
{
	$project: {
		name: "$employee_info.name",
		surname: "$employee_info.surname",
		_id: 0,
		count: 1,
		monetary_turnover: 1
	}
}
])
\end{lstlisting}

\newpage
\section{Dyskusja zrealizowanych technik}

\subsection{Trigger przy dodawaniu nowej rezerwacji}
Podczas dodawania nowej rezerwacji, by nie przechowywać zbędnych starych rezerwacji, usuwane są wszystkie rezerwacje, których data jest wcześniejsza niż aktualny dzień. CleanReservations to funkcja, która odpowiada za wyszukanie i usunięcie starych rezerwacji
\begin{lstlisting}[caption={Trigger}]
	resSchema.post('save', function(doc) {
		cleanReservations();
	})
\end{lstlisting}

\subsection{Możliwość zapisu raportu}
Nasza aplikacja wspiera zapisywanie wygenerowanego raportu do pliku .csv, który można potem łatwo analizować i obrabiać. Zdecydowaliśmy się na to, ponieważ uważamy, że samo wyświetlanie się danych nie byłoby wystarczające dla klienta i na pewno takie usprawnienie zwiększa wygodę używania aplikacji.

todo: jescze jakies jedno by sie przydalo

\end{document}